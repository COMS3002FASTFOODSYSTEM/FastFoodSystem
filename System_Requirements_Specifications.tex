%%%%%%%%%%%%%%%%%%%%%%%%%%%%%%%%%%%%%%%%%
% Structured General Purpose Assignment
% LaTeX Template
%
% This template has been downloaded from:
% http://www.latextemplates.com
%
% Original author:
% Ted Pavlic (http://www.tedpavlic.com)
%
% Note:
% The \lipsum[#] commands throughout this template generate dummy text
% to fill the template out. These commands should all be removed when 
% writing assignment content.
%
%%%%%%%%%%%%%%%%%%%%%%%%%%%%%%%%%%%%%%%%%

%----------------------------------------------------------------------------------------
%	PACKAGES AND OTHER DOCUMENT CONFIGURATIONS
%----------------------------------------------------------------------------------------

\documentclass{article}

\usepackage{fancyhdr} % Required for custom headers
\usepackage{lastpage} % Required to determine the last page for the footer
\usepackage{extramarks} % Required for headers and footers
\usepackage{graphicx} % Required to insert images
\usepackage{lipsum} % Used for inserting dummy 'Lorem ipsum' text into the template

% Margins
\topmargin=-0.45in
\evensidemargin=0in
\oddsidemargin=0in
\textwidth=6.5in
\textheight=9.0in
\headsep=0.25in 

\linespread{1.1} % Line spacing

% Set up the header and footer
\pagestyle{fancy}
\lhead{\hmwkAuthorName} % Top left header
\chead{\hmwkClass} % Top center header
\rhead{\firstxmark} % Top right header
\lfoot{\lastxmark} % Bottom left footer
\cfoot{} % Bottom center footer
\rfoot{Page\ \thepage\ of\ \pageref{LastPage}} % Bottom right footer
\renewcommand\headrulewidth{0.4pt} % Size of the header rule
\renewcommand\footrulewidth{0.4pt} % Size of the footer rule

\setlength\parindent{0pt} % Removes all indentation from paragraphs

%----------------------------------------------------------------------------------------
%	DOCUMENT STRUCTURE COMMANDS
%	Skip this unless you know what you're doing
%----------------------------------------------------------------------------------------

% Header and footer for when a page split occurs within a problem environment
\newcommand{\enterProblemHeader}[1]{
\nobreak\extramarks{#1}{#1 continued on next page\ldots}\nobreak
\nobreak\extramarks{#1 (continued)}{#1 continued on next page\ldots}\nobreak
}

% Header and footer for when a page split occurs between problem environments
\newcommand{\exitProblemHeader}[1]{
\nobreak\extramarks{#1 (continued)}{#1 continued on next page\ldots}\nobreak
\nobreak\extramarks{#1}{}\nobreak
}

\setcounter{secnumdepth}{0} % Removes default section numbers
\newcounter{homeworkProblemCounter} % Creates a counter to keep track of the number of problems

\newcommand{\homeworkProblemName}{}
\newenvironment{homeworkProblem}[1][Problem \arabic{homeworkProblemCounter}]{ % Makes a new environment called homeworkProblem which takes 1 argument (custom name) but the default is "Problem #"
\stepcounter{homeworkProblemCounter} % Increase counter for number of problems
\renewcommand{\homeworkProblemName}{#1} % Assign \homeworkProblemName the name of the problem
\section{\homeworkProblemName} % Make a section in the document with the custom problem count
\enterProblemHeader{\homeworkProblemName} % Header and footer within the environment
}{
\exitProblemHeader{\homeworkProblemName} % Header and footer after the environment
}

\newcommand{\problemAnswer}[1]{ % Defines the problem answer command with the content as the only argument
\noindent\framebox[\columnwidth][c]{\begin{minipage}{0.98\columnwidth}#1\end{minipage}} % Makes the box around the problem answer and puts the content inside
}

\newcommand{\homeworkSectionName}{}
\newenvironment{homeworkSection}[1]{ % New environment for sections within homework problems, takes 1 argument - the name of the section
\renewcommand{\homeworkSectionName}{#1} % Assign \homeworkSectionName to the name of the section from the environment argument
\subsection{\homeworkSectionName} % Make a subsection with the custom name of the subsection
\enterProblemHeader{\homeworkProblemName\ [\homeworkSectionName]} % Header and footer within the environment
}{
\enterProblemHeader{\homeworkProblemName} % Header and footer after the environment
}
   
%----------------------------------------------------------------------------------------
%	NAME AND CLASS SECTION
%----------------------------------------------------------------------------------------

\newcommand{\hmwkTitle}{System Requirements Specifications} % Assignment title
\newcommand{\hmwkDueDate}{Monday,\ August\ 28,\ 2017} % Due date
\newcommand{\hmwkClass}{Sofware Engineering} % Course/class
\newcommand{\hmwkClassTime}{COMS3002} % Class/lecture time
\newcommand{\hmwkClassInstructor}{Daniel} % Teacher/lecturer
\newcommand{\hmwkAuthorName}{Group 4} % Your name

%----------------------------------------------------------------------------------------
%	TITLE PAGE
%----------------------------------------------------------------------------------------

\title{
\vspace{2in}
\textmd{\textbf{\hmwkClass:\ \hmwkTitle}}\\
\normalsize\vspace{0.1in}\small{Due\ on\ \hmwkDueDate}\\
\vspace{0.1in}\large{\textit{\hmwkClassInstructor\ \hmwkClassTime}}
\vspace{3in}
}

\author{\textbf{\hmwkAuthorName}}
\date{} % Insert date here if you want it to appear below your name

%----------------------------------------------------------------------------------------

\begin{document}

\maketitle

%----------------------------------------------------------------------------------------
%	TABLE OF CONTENTS
%----------------------------------------------------------------------------------------

\setcounter{tocdepth}{1} % Uncomment this line if you don't want subsections listed in the ToC

\newpage
\tableofcontents
\newpage

%----------------------------------------------------------------------------------------
%	PROBLEM 1
%----------------------------------------------------------------------------------------

% To have just one problem per page, simply put a \clearpage after each problem

\begin{homeworkProblem}[Introduction]
	\begin{homeworkSection}{Purpose} %/ use this block to create subsections
	
	\end{homeworkSection}

\end{homeworkProblem}

%----------------------------------------------------------------------------------------
%	PROBLEM 2
%----------------------------------------------------------------------------------------

\begin{homeworkProblem}[Overall Description]
	\begin{homeworkSection}{Product perspective} %/ use this block to create subsections
	The product being represented in this document is a Fast Food Ordering System, a first of it's kind. The software will take the form of a three-tier client-server architecture, with it's primary aim being to give consumers the flexibility of ordering from any restaurant or fast-food outlet. The software will save consumers the time and effort of having to wait in long queues to order their food. The orders are to be made and paid for one online, with the option of choosing to eat at the restaurant or simply to collect your food to eat in the comfort of your own home.
	
	% one last paragraph about security.
	\end{homeworkSection}
	
	\begin{homeworkSection}{Product functions} %/ use this block to create subsections
		\begin{itemize}
 		 \item Enable the user to log on to the system so that orders can be tracked
  		\item Make use of geo-location services to give the user recommendations on places to eat based on a 20km radius
  		\item Have a user friendly menu to avoid confusion when ordering food
  		\item When an order has been made, give a the user a period of 10 minutes to make changes to the order
  		\item Make online payment possible via credit/debit card
  		\item Make online payment secure
  		\item Make use of geo-location services to give user directions to the restaurant 
  		\item Keep track of all the orders coming through using  numbers 
  		
		\end{itemize}
	\end{homeworkSection}

	\begin{homeworkSection}{User Classes and Characteristics} %/ use this block to create 		subsections
	From the consumer side, anyone with access to a smart-phone and a working bank account is a potential user of the software. To narrow it down a bit, it will be well suited for the working class, as they have long working hours and they are often stuck in traffic. Thus this software will be perfect for them, i.e they can order whilst they are stuck in traffic, and by the time they get to the food outlet their order will be ready to take home. 
	
	Looking at the service providers (restaurants and fast food outlets), the system will be used by trained staff members. Training which is to be received upon installation of the software.
	\end{homeworkSection}

	
	\begin{homeworkSection}{Operating Environment} %/ use this block to create subsections
	This software will be designed to run in an internet browser, as a result it will be accessible to most operating systems. It will be developed in conjunction with the Google API to provide most of it's geo-location services. PostgreSQL will be the open source database management system that will be in place to implement all relationships that exist between datasets.
	
	\end{homeworkSection}
	
	\begin{homeworkSection}{Design and Implementation Constraints} %/ use this block to create subsections
	Ehhh Leaders, can I please be given a hand here!!!!!!!
	\end{homeworkSection}
	
	
	\begin{homeworkSection}{User Documentation} %/ use this block to create subsections
	
	\begin{itemize}
	\item Consumers will be given a comprehensive user-manual
	\item Services providers will also be given their own user-manual which will also double as training manual.
	\end{itemize}
	\end{homeworkSection}

\end{homeworkProblem}



\begin{homeworkProblem}[External Interface Requirements]
	\begin{homeworkSection}{User Interfaces} %/ copy this block to create subsections
	The system may later be adapted to work as a desktop website but currently we are only focusing on the mobile versions, iOS, Android as well as Windows mobile. That being said, the main feel of the app will be a modern crossplatform design so tiles and buttons will be of AngularJS, CSS3 as well as Ionic.

	The first page is a Welcome screen then a Home screen. Options that stay available on each screen is an option icon with:
		\begin{itemize}
		\item Preferences
		\item Help
		\item Feedback
		\item About
		\item Logout
		\end{itemize}
	The Footer of the App will have 3 fixed buttons (more like Instagram). The buttons shall respecitively be: 
		\begin{itemize}
		\item Orders
		\item Home
		\item (Idle)
		\end{itemize}
	\end{homeworkSection}
	\begin{homeworkSection}{Hardware Interfaces} %/ copy this block to create subsections
	Since by design the App is crossplaform the constraints here needed to be a bit more flexible, however GPS location as well as mobile identity permissions will be required. The app should run in any Android device of version 6.0 or above, iOSX and Windows 10 mobile.
	\end{homeworkSection}
	\begin{homeworkSection}{Software Interfaces} %/ copy this block to create subsections
	An SQL database management system, preferably PostgreSQL shall be used for this three-tier system. The Google Maps API should be used to make it easier for users to visualise their locations as well as locations of the local restuarents. 
	\end{homeworkSection}
	\begin{homeworkSection}{Communication Interfaces} %/ copy this block to create subsections
	The main protocal of communication will be HTTP. So we expect encrypted data to shared as JSON arrays or anything better. Libraries like that of JQuery should be considered when handling form inputs that have been sanitized. Minimize response time at all times so persistant open threads may need to be open on the server-side.
	\end{homeworkSection}

\end{homeworkProblem}


\begin{homeworkProblem}[Functional Requirements: System Features]
	\begin{homeworkSection}{System Feature 1} %/ copy this block to create subsections
	
	\end{homeworkSection}

\end{homeworkProblem}


\begin{homeworkProblem}[Non-functional Requirements]
	\begin{homeworkSection}{Performance Requirements} %/ use this block to create subsections
	
	\end{homeworkSection}

\end{homeworkProblem}


%----------------------------------------------------------------------------------------

\end{document}
