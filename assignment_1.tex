%%%%%%%%%%%%%%%%%%%%%%%%%%%%%%%%%%%%%%%%%
% Structured General Purpose Assignment
% LaTeX Template
%
% This template has been downloaded from:
% http://www.latextemplates.com
%
% Original author:
% Ted Pavlic (http://www.tedpavlic.com)
%
% Note:
% The \lipsum[#] commands throughout this template generate dummy text
% to fill the template out. These commands should all be removed when 
% writing assignment content.
%
%%%%%%%%%%%%%%%%%%%%%%%%%%%%%%%%%%%%%%%%%

%----------------------------------------------------------------------------------------
%	PACKAGES AND OTHER DOCUMENT CONFIGURATIONS
%----------------------------------------------------------------------------------------

\documentclass{article}

\usepackage{fancyhdr} % Required for custom headers
\usepackage{lastpage} % Required to determine the last page for the footer
\usepackage{extramarks} % Required for headers and footers
\usepackage{graphicx} % Required to insert images
\usepackage{lipsum} % Used for inserting dummy 'Lorem ipsum' text into the template

% Margins
\topmargin=-0.45in
\evensidemargin=0in
\oddsidemargin=0in
\textwidth=6.5in
\textheight=9.0in
\headsep=0.25in 

\linespread{1.1} % Line spacing

% Set up the header and footer
\pagestyle{fancy}
\lhead{\hmwkAuthorName} % Top left header

\rhead{\firstxmark} % Top right header
\lfoot{\lastxmark} % Bottom left footer
\cfoot{} % Bottom center footer
\rfoot{Page\ \thepage\ of\ \pageref{LastPage}} % Bottom right footer
\renewcommand\headrulewidth{0.4pt} % Size of the header rule
\renewcommand\footrulewidth{0.4pt} % Size of the footer rule

\setlength\parindent{0pt} % Removes all indentation from paragraphs

%----------------------------------------------------------------------------------------
%	DOCUMENT STRUCTURE COMMANDS
%	Skip this unless you know what you're doing
%----------------------------------------------------------------------------------------

% Header and footer for when a page split occurs within a problem environment
\newcommand{\enterProblemHeader}[1]{
\nobreak\extramarks{#1}{#1 continued on next page\ldots}\nobreak
\nobreak\extramarks{#1 (continued)}{#1 continued on next page\ldots}\nobreak
}

% Header and footer for when a page split occurs between problem environments
\newcommand{\exitProblemHeader}[1]{
\nobreak\extramarks{#1 (continued)}{#1 continued on next page\ldots}\nobreak
\nobreak\extramarks{#1}{}\nobreak
}

\setcounter{secnumdepth}{0} % Removes default section numbers
\newcounter{homeworkProblemCounter} % Creates a counter to keep track of the number of problems

\newcommand{\homeworkProblemName}{}
\newenvironment{homeworkProblem}[1][Problem \arabic{homeworkProblemCounter}]{ % Makes a new environment called homeworkProblem which takes 1 argument (custom name) but the default is "Problem #"
\stepcounter{homeworkProblemCounter} % Increase counter for number of problems
\renewcommand{\homeworkProblemName}{#1} % Assign \homeworkProblemName the name of the problem
\section{\homeworkProblemName} % Make a section in the document with the custom problem count
\enterProblemHeader{\homeworkProblemName} % Header and footer within the environment
}{
\exitProblemHeader{\homeworkProblemName} % Header and footer after the environment
}

\newcommand{\problemAnswer}[1]{ % Defines the problem answer command with the content as the only argument
\noindent\framebox[\columnwidth][c]{\begin{minipage}{0.98\columnwidth}#1\end{minipage}} % Makes the box around the problem answer and puts the content inside
}

\newcommand{\homeworkSectionName}{}
\newenvironment{homeworkSection}[1]{ % New environment for sections within homework problems, takes 1 argument - the name of the section
\renewcommand{\homeworkSectionName}{#1} % Assign \homeworkSectionName to the name of the section from the environment argument
\subsection{\homeworkSectionName} % Make a subsection with the custom name of the subsection
\enterProblemHeader{\homeworkProblemName\ [\homeworkSectionName]} % Header and footer within the environment
}{
\enterProblemHeader{\homeworkProblemName} % Header and footer after the environment
}
   
%----------------------------------------------------------------------------------------
%	NAME AND CLASS SECTION
%----------------------------------------------------------------------------------------

\newcommand{\hmwkTitle}{Assignment\ \#1} % Assignment title
\newcommand{\hmwkDueDate}{Monday,\ January\ 1,\ 2012} % Due date
\newcommand{\hmwkClass}{BIO\ 101} % Course/class
\newcommand{\hmwkClassTime}{10:30am} % Class/lecture time
\newcommand{\hmwkClassInstructor}{Daniel Holmes} % Teacher/lecturer
\newcommand{\hmwkAuthorName}{Group 4 : Fast-Food Management System} % Your name

%----------------------------------------------------------------------------------------
%	TITLE PAGE
%----------------------------------------------------------------------------------------

\title{
\vspace{2in}
\textmd{\textbf{COMS3002 : Software Engineering}}\\
\normalsize\vspace{0.1in}\small{Due\ on\ Monday August 14 2017}\\
\vspace{0.1in}\large{\textit{\hmwkClassInstructor}}
\vspace{3in}
}

\author{\textbf{\hmwkAuthorName}}
\date{} % Insert date here if you want it to appear below your name

%----------------------------------------------------------------------------------------

\begin{document}

\maketitle

%----------------------------------------------------------------------------------------
%	TABLE OF CONTENTS
%----------------------------------------------------------------------------------------

%\setcounter{tocdepth}{1} % Uncomment this line if you don't want subsections listed in the ToC

\newpage
\tableofcontents
\newpage

%----------------------------------------------------------------------------------------
%	PROBLEM 1
%----------------------------------------------------------------------------------------

% To have just one problem per page, simply put a \clearpage after each problem

\begin{homeworkProblem}[Overview]
\vspace{10pt} % Question

\problemAnswer{ % Answer
We are going to build a web application to facilitate making orders at selected Fast-Food restaurants. Our application will allow users to sign-up an account, select a restaurant within a certain radius of their geo-location and upon selecting their order they may choose to eat-in at the restaurant or pick up their order. An order is consider complete after payment has been settled using a unique reference number issued when the order was placed. We will ensure that all payments are secured and customer credentials protected so that no fraudulent activities may occur.
}
\end{homeworkProblem}

%----------------------------------------------------------------------------------------
%	PROBLEM 2
%----------------------------------------------------------------------------------------

\begin{homeworkProblem}[Objectives and Goals] 

%--------------------------------------------

-Enable users to create an account using their email address to log their activity.

-Store payment credentials together with the user account

-Make online payment secure

-Login accomplished by entering username(email) and password.

-Use Geo-location to isolate near-by Fast-food restuarents.

-Restuarants menu to be displayed in-app 

-Create easy to navigate menus

-A shopping basket implementation to store selected items.

-Order reciept to be generated and sent to customer for book-keeping.
%--------------------------------------------

\end{homeworkProblem}

%----------------------------------------------------------------------------------------
%	PROBLEM 3
%----------------------------------------------------------------------------------------

\begin{homeworkProblem}[Responsibilities] % Roman numerals

%--------------------------------------------

\begin{homeworkSection}{Front-End : Garfield Moganedi and Kamogelo Khamiwa} % Using the problem name elsewhere

-HTML,CSS and JavaScript implementation of our user interface

-Aesthetic layout of Website

-Write modularized code for optimisation and reusable code.

-Incoporate restuarant menus into display

-Coherent navigation of internet application

-Ensure feasibility in user interface design so it is responsive.

-Ensure all user input is validated before passing it to the back end.

-Provide map to diplay Geo-location of selected location. 

\end{homeworkSection}

%--------------------------------------------

\begin{homeworkSection}{Back-End : Nthikeng Letsoalo and Mohale Nakana}
-Structuring of databases(Restuarants, User Accounts) for data storage solutions.

-Logical handling of order inputs and delivering relevent output information.

-Integration of user facing elements with server logic.

-Implement data security to protect sensitive user input.

-Use open GIS services to get location information

-Implement secure payment facilities.


\end{homeworkSection}

%--------------------------------------------

\end{homeworkProblem}

%----------------------------------------------------------------------------------------
%	PROBLEM 4
%----------------------------------------------------------------------------------------

\begin{homeworkProblem}[Expected Input/Output] % Roman numerals
\begin{homeworkSection}{Registration}
-First Name

-Last Name

-Email Address

-Password

\problemAnswer{
Our output will be confirmation of a registered account
}

\end{homeworkSection}

\begin{homeworkSection}{Order page}
-Geo-location

-Preferred location

\problemAnswer{
Our output is the available options within the selected area
}


\end{homeworkSection}

\begin{homeworkSection}{Placing an Order}
-Selected restuarant

-Selected items

-Method of collection (Eating-in or Collecting)

\problemAnswer{
Our output is the customers Order number.
}

\end{homeworkSection}


\begin{homeworkSection}{Payment page}
-Card number

-CVD number

\problemAnswer{
Our output is the confirmation of your payment and a receipt is issued with the clients unique reference number. A map showing the directions to the location is then displayed 
}

\end{homeworkSection}

\end{homeworkProblem}

%----------------------------------------------------------------------------------------

\end{document}
