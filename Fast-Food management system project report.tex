%%%%%%%%%%%%%%%%%%%%%%%%%%%%%%%%%%%%%%%%%
% Structured General Purpose Assignment
% LaTeX Template
%
% This template has been downloaded from:
% http://www.latextemplates.com
%
% Original author:
% Ted Pavlic (http://www.tedpavlic.com)
%
% Note:
% The \lipsum[#] commands throughout this template generate dummy text
% to fill the template out. These commands should all be removed when 
% writing assignment content.
%
%%%%%%%%%%%%%%%%%%%%%%%%%%%%%%%%%%%%%%%%%

%----------------------------------------------------------------------------------------
%	PACKAGES AND OTHER DOCUMENT CONFIGURATIONS
%----------------------------------------------------------------------------------------

\documentclass{article}
\usepackage{graphicx} 
\graphicspath{ {C:\Users\KC\Desktop\images} }

\usepackage{fancyhdr} % Required for custom headers
\usepackage{lastpage} % Required to determine the last page for the footer
\usepackage{extramarks} % Required for headers and footers
\usepackage{graphicx} % Required to insert images
\usepackage{lipsum} % Used for inserting dummy 'Lorem ipsum' text into the template

% Margins
\topmargin=-0.45in
\evensidemargin=0in
\oddsidemargin=0in
\textwidth=6.5in
\textheight=9.0in
\headsep=0.25in 

\linespread{1.1} % Line spacing

% Set up the header and footer
\pagestyle{fancy}
\lhead{\hmwkAuthorName} % Top left header
\chead{\hmwkClass} % Top center header
\rhead{\firstxmark} % Top right header
\lfoot{\lastxmark} % Bottom left footer
\cfoot{} % Bottom center footer
\rfoot{Page\ \thepage\ of\ \pageref{LastPage}} % Bottom right footer
\renewcommand\headrulewidth{0.4pt} % Size of the header rule
\renewcommand\footrulewidth{0.4pt} % Size of the footer rule

\setlength\parindent{0pt} % Removes all indentation from paragraphs

%----------------------------------------------------------------------------------------
%	DOCUMENT STRUCTURE COMMANDS
%	Skip this unless you know what you're doing
%----------------------------------------------------------------------------------------

% Header and footer for when a page split occurs within a problem environment
\newcommand{\enterProblemHeader}[1]{
\nobreak\extramarks{#1}{#1 continued on next page\ldots}\nobreak
\nobreak\extramarks{#1 (continued)}{#1 continued on next page\ldots}\nobreak
}

% Header and footer for when a page split occurs between problem environments
\newcommand{\exitProblemHeader}[1]{
\nobreak\extramarks{#1 (continued)}{#1 continued on next page\ldots}\nobreak
\nobreak\extramarks{#1}{}\nobreak
}

\setcounter{secnumdepth}{0} % Removes default section numbers
\newcounter{homeworkProblemCounter} % Creates a counter to keep track of the number of problems

\newcommand{\homeworkProblemName}{}
\newenvironment{homeworkProblem}[1][Problem \arabic{homeworkProblemCounter}]{ % Makes a new environment called homeworkProblem which takes 1 argument (custom name) but the default is "Problem #"
\stepcounter{homeworkProblemCounter} % Increase counter for number of problems
\renewcommand{\homeworkProblemName}{#1} % Assign \homeworkProblemName the name of the problem
\section{\homeworkProblemName} % Make a section in the document with the custom problem count
\enterProblemHeader{\homeworkProblemName} % Header and footer within the environment
}{
\exitProblemHeader{\homeworkProblemName} % Header and footer after the environment
}

\newcommand{\problemAnswer}[1]{ % Defines the problem answer command with the content as the only argument
\noindent\framebox[\columnwidth][c]{\begin{minipage}{0.98\columnwidth}#1\end{minipage}} % Makes the box around the problem answer and puts the content inside
}

\newcommand{\homeworkSectionName}{}
\newenvironment{homeworkSection}[1]{ % New environment for sections within homework problems, takes 1 argument - the name of the section
\renewcommand{\homeworkSectionName}{#1} % Assign \homeworkSectionName to the name of the section from the environment argument
\subsection{\homeworkSectionName} % Make a subsection with the custom name of the subsection
\enterProblemHeader{\homeworkProblemName\ [\homeworkSectionName]} % Header and footer within the environment
}{
\enterProblemHeader{\homeworkProblemName} % Header and footer after the environment
}
   
%----------------------------------------------------------------------------------------
%	NAME AND CLASS SECTION
%----------------------------------------------------------------------------------------

\newcommand{\hmwkTitle}{Fast-Food management system project report} % Assignment title
\newcommand{\hmwkDueDate}{Monday,\ October\ 2\ 2017} % Due date
\newcommand{\hmwkClass}{Sofware Engineering} % Course/class
\newcommand{\hmwkClassTime}{COMS3002} % Class/lecture time
\newcommand{\hmwkClassInstructor}{Daniel Holmes} % Teacher/lecturer
\newcommand{\hmwkAuthorName}{Group 4} % Your name

%----------------------------------------------------------------------------------------
%	TITLE PAGE
%----------------------------------------------------------------------------------------

\title{
\vspace{2in}
\textmd{\textbf{\hmwkClass:\ \hmwkTitle}}\\
\normalsize\vspace{0.1in}\small{Due\ on\ \hmwkDueDate}\\
\vspace{0.1in}\large{\textit{\hmwkClassInstructor\ \hmwkClassTime}}
\vspace{3in}
}

\author{\textbf{\hmwkAuthorName}}
\date{} % Insert date here if you want it to appear below your name

%----------------------------------------------------------------------------------------

\begin{document}

\maketitle

%----------------------------------------------------------------------------------------
%	TABLE OF CONTENTS
%----------------------------------------------------------------------------------------

\setcounter{tocdepth}{1} % Uncomment this line if you don't want subsections listed in the ToC

\newpage
\tableofcontents
\newpage

%----------------------------------------------------------------------------------------
%	PROBLEM 1
%----------------------------------------------------------------------------------------

% To have just one problem per page, simply put a \clearpage after each problem

\begin{homeworkProblem}[Introduction]
	\begin{homeworkSection}{Project Overview} %/ use this block to create subsections
		Fast Food Management System brings the people to food and bye-bye to queues in Fast food restaurants. Registered fast food restaurants will be alleviated of queues by receiving orders online and providing users with estimate preparation time. Customers will then collect the food at the establishment or eat-in. 
	
	\end{homeworkSection}
	
	\begin{homeworkSection}{Intended Audience and Reading Suggestions}
	
	Software development teams, our peers and superiors alike are who this document is addressed to. Now, this document is written in an effort to facilitate cross-departmental contribution to this project. We open with an overall description to state our intended end product followed by external interface requirements to expand upon our development strategies. System features are included to further modularize the necessary tasks our project must undertake. Nonfunctional requirements explain our intended performance security goals. 
	\end{homeworkSection}
	
	\begin{homeworkSection}{Product Scope}
	\textbf{Current Situation}:
	
	Similar services are being provided by Square Order, Grub-Hub and eHungry.
	These applications provide users the means to place orders online through a
	marketspace site implementation that is a platform on which restaurants can
	host their mobile ordering functionality. We intend on creating an application
	that maintains the individualism of each restaurant. Our application is
	customer oriented in that instead of pooling all our registered restaurants we
	will present the most convenient establishments based on the customers
	geo-location. 
	
	\textbf{}
	
	\textbf{Work Partitioning}:
	
	\begin{itemize}
	\item \textbf{Sign Up}: 
	    \begin{itemize}
	        \item INPUT:
	        
	        Customer enters their email address and password which will be stored
	        in the user database.
	        \item OUTPUT:
	        
	        Once successfully stored in our database the customer will now have an
	        active account through which they may access the application
	        functionality.
	    \end{itemize}
	    
	\item \textbf{Sign In}: 
	    \begin{itemize}
	        \item INPUT:
	        
	        Customer enters their email address and password which will be checked
	        in the user database as to whether the user exists.
	        \item OUTPUT:
	        
	        Once successfully validated in our database the customer will now be
	        taken to their Home page where they may choose to log-out,enter
	        administration services (Restaurant only) or initiate the restaurant
	        locator and proceed to order food.
	    \end{itemize}
	    
	 \item \textbf{Restaurant Locator}: 
	    \begin{itemize}
	        \item INPUT:
	        
	        Geo-location of the customer's device location is gathered and
	        transmitted to our server.
	        \item OUTPUT:
	        
	        Based on proximity information the application will provide the user
	        with a list of near-by restaurants which, once one is selected, will
	        lead the customer to the ordering menu of their selected establishment.
	    \end{itemize}
	    
	\item \textbf{Order Booking}: 
	    \begin{itemize}
	        \item INPUT:
	        
	        Customer enters their menu selections and proceed to place their order.
	        The order is received by our back-end and payment must be made before
	        finalizing the order.
	        \item OUTPUT:
	        
	        Once an order is successfully placed the customer is issued a unique
	        reference number which will be required by the establishment to collect
	        their order and complete the transaction.
	    \end{itemize}    
	
	\end{itemize}
	
	\end{homeworkSection}
	
	\begin{homeworkSection}{Definitions}
	\textbf{Three-tier client-server architecture} : System Architecture that will define three logically independent tiers, namely, Presentation, Data management and Domain logic.
	
\textbf{Geo-location} : An estimation of real world geographical location.

\textbf{Service providers} : Refers to restaurants and onther registered fast food outlets

\textbf{PostgreSQL} : An open source Database management system.

\textbf{iOS(X)} : Operating system used by mobile Apple devices.

\textbf{Android} : Google's Most common open-source operating system used by mobile devices.

\textbf{Windows 10 mobile} : Windows operating system ported for mobile device functionality.

\textbf{AngularJS} : A JavaScript framework that allows a dynamic page refresh.

\textbf{CSS3} : A framework used to format webpage layout.

\textbf{Ionic} : A modern HTML5, CSS framework that supports mobile application development.

\textbf{JSON} : (JavaScript Object Notation) Is used to transmit data between server and web application.

\textbf{JQuery} : JavaScript framework that allows manipulation of a web page to enble user interaction.

	\end{homeworkSection}
	\begin{homeworkSection}{References}
	
	\textbf{- van Vliet, H, (2007), Software Engineering Principles and Practice,Wiley.}
	
	
	\end{homeworkSection}
	
\end{homeworkProblem}

%----------------------------------------------------------------------------------------
%	PROBLEM 2
%----------------------------------------------------------------------------------------

\begin{homeworkProblem}[Overall Description]
	\begin{homeworkSection}{Product perspective} %/ use this block to create subsections
    The application will take the form of a three-tier client-server architecture. On accessing the application, customers are able to place orders at their selected establishment. Geo-location information is gathered and fast food restaurants will appear in order of proximity from our customers recorded geo-location. Menu presentation of the selected restaurant is followed by a place order interface the customer will use to select their desired items. Payment credentials will be secured and stored prior to issuing an order number to our customer which must be presented on collection of their order. Administrator functionality will include the option to alter/delete menu presentations with store and review functionality for customer orders, this is necessary for monthly reports.
	
	% one last paragraph about security.
	\end{homeworkSection}
	
    \begin{homeworkSection}{Stakeholders}
    
        \begin{itemize}
            \item \textbf{Developers}:
                \begin{itemize}
                    \item \textbf{Front-end}:
                    Front-end developers will be using Android Studio and Java to create the aesthetics of the application
                    \item\textbf{Back-end}:
                    Back-end developers will be using the online Firebase database Console to maintain data integrity throughout the system for our Customers and Clients.
                \end{itemize}
        
            \item \textbf{The Client}:
            Our clientele will be Fast-Fast restaurant industry. We provide a restaurant with the means to optimize their service delivery by providing a means to expand their customer base. Not only will residents of a particular location be potential customers but now anyone with the application could be a customer.
            
            \item \textbf{Hands-on users of the product}:
            Our customers are the fast-food consumers who need efficient preparation and delivery of placed orders.
            
                \textbf{Admin} :
                \begin{itemize}
                  \item Admin can view orders made by customers
                  \item Can change menu of their restaurant 
                  \item They can also give feed back
                \end{itemize}
                    \textbf{ Customers} :
                \begin{itemize}
                 \item Customers can create an account
                 \item Customers can make an order, they can cancel an order (before the delay time have not elapsed)
                 \item They can give feedback     
                 \item They can also specify a restaurant they want to place on order to
                 \end{itemize} 
         
         The only training necessary is for admin users who will need basic computer literacy.
            \item \textbf{Priorities assigned to users}:
            
                \textbf{ Customers} : Have the least privileged permissions. Customers need only interact with the database passively when signing up and Logging in to the application, their priority is to order food and receive a reference number for order collection.
                
                \textbf{ Admin} :Have the most privileged permissions. Admin reserves the right to cancel an order on non-collection and may change their menu selections. Admin priorities are limited to updating the menu selections, acknowledging incoming orders and preparing the customers order.
        \end{itemize}
    
    \end{homeworkSection}
    
    \begin{homeworkSection}{Mandated Constraints}
    
    This section describes constraints on the eventual design of the product. They are the same as other requirements except that constraints were mandated at the beginning of the project.
        \begin{itemize}
            \item Pick-up location must be selected by the client.
            \item There must be a time period allowed for changes to a customers order.
            \item The service is only usable for customers that issue orders within 20km of a pick-up or eat-in restaurant.
        \end{itemize}
    
    \end{homeworkSection}
    
    \begin{homeworkSection}{Solution Constraints}
        \begin{itemize}
            \item High accuracy GPS is very crucial for the system to work at it's best. Also the app should not use more than 128MB of the RAM and take minimum CPU time as that may hamper smooth operation of the entire smartphone.
            
            \item HTTP will be our communication protocol. We expect encrypted data to shared as JSON arrays. We want to minimize response time so persistent open threads will need to be available on the server-side.
        \end{itemize}
    
    \end{homeworkSection}
    
    \begin{homeworkSection}{Implementation Environment of the current system}
    
    This software is designed to run on android smartphones. It shall be developed using Android Studio IDE in java, in conjunction with the Google API to provide for geographic-location services. Firebase will be the open source database management system that will be in place to implement all relationships that exist between datasets.
    
    \end{homeworkSection}
    
    \begin{homeworkSection}{User Documentation} %/ use this block to create subsections
	
	\begin{itemize}
	\item Consumers will be given a 1-page comprehensive graphical user-manual
	\item Clients will also be given their own user-manual which will also be used as a staff training manual
	\end{itemize}
	\end{homeworkSection}
    
    \begin{homeworkSection}{Proposed Software Architecture}
    System Architecture that will define three logically independent tiers, namely, Presentation, Data management and Domain logic.
    
    \begin{itemize}
        \item \textbf{Firebase Database}
        The database communicates with the application using JSON arrays sent using HTTPS. We use it to validate user login and store user information after sign up. It returns restaurant location and menu details to the Customer-side app and delivers customer order details to the Client-side app
        \begin{itemize}
            \item \textbf{Tables}:
            \begin{itemize}
                \item Users:
                
                -Two columns, User email and Password.
                
                -User email address will be used as a Primary key.
                
                -A user Profile object will be created using these values
                
                \item Restaurants:
                
                -Two columns, Restaurant location and Restaurant menu
                
                -Restaurant location will be used as a primary key
                
                -A restaurant ordering object will be created on the customer app using these values.
            \end{itemize}
        \end{itemize}
    \end{itemize}
    
    \end{homeworkSection}
	
\end{homeworkProblem}



\begin{homeworkProblem}[External Interface Requirements]
	\begin{homeworkSection}{Customer Interfaces} %/ copy this block to create subsections
	Currently we are only focusing on the mobile versions, Android mobile in particular. That being said, the main feel of the app will be a modern design so tiles and buttons will be constructed using XML & HTML5.
	
	The first page is a Home screen. The footer of the app will have 3 fixed buttons: 
		\begin{itemize}
		\item Order - To show pending orders and well as approved orders
		\item Logout - To go to the Home screen
		\item (Idle) - This button on the footer doesn't have a function as of yet
		\end{itemize}
	\end{homeworkSection}
	\begin{homeworkSection}{Hardware Interfaces} %/ copy this block to create subsections
	 The app should run on any Android device running API 26 or later.
	\end{homeworkSection}
	\begin{homeworkSection}{Software Interfaces} %/ copy this block to create subsections
	A noSQL database management system,Firebase shall be used for this three-tier system. The Google Maps API should be used for users to visualize their locations as well as locations of the local restaurants. 
	\end{homeworkSection}

\end{homeworkProblem}


\begin{homeworkProblem}[Functional Requirements: System Features]
	\begin{homeworkSection}{Pages and permissions} %/ copy this block to create subsections
	\begin{itemize}	
	\item Restaurant locator
	\end{itemize}

 	- Identify restuarant name
 
 	- Calculate distance to restaurant

 	- Connect to restaurant server

 	- Load admin/customer interface

	\begin{itemize}
	\item Admin
	\end{itemize}

 	- Can view and change the restaurant menu 
 	
 	- Order coordination to ensure effiecient meal preparation for shortest waiting times
 	
 	- Can view user accounts and see accumulated reciepts and log

	\begin{itemize}
	\item Customer 
	\end{itemize}

	- Places an order online

	- Selects preferred restaurant and places order

	- Payment credentials are stored and protected

	- Upon successful placement of order, a unique reference number is given to the customer. The number shall be used to access the customers payment credentials

	- Customer selects preparation method, eat-in/collect. Their order is then verified and at this point the customer may change their order

	- Order is verified and stored in archive. Approximate preperation time and map route(address) is displayed
	
	\begin{itemize}
	\item Customer locator
	\end{itemize}
	
	- Identify customer�s location

	- Retrieve proxity information from restaurant server and display restaurant location to customer and route.

	\end{homeworkSection}

	\begin{homeworkSection}{Logistic management}

	\begin{itemize}
	\item User accounts
	\end{itemize}

	- Users can register an account that will need their email and password in this way customers can store their receipts with us

	\begin{itemize}
	\item Menu display
	\end{itemize}

	- Restaurant menus need to be quite presantable

	- The order placing interface will comprise of drop box item selections and view of accumulated cost
	
	\begin{itemize}
	\item Unique reference number
	\end{itemize}
	
	\end{homeworkSection}
	
	

\end{homeworkProblem}


\begin{homeworkProblem}[Non-functional Requirements]
		
	\begin{homeworkSection}{Performance Requirements} %/ use this block to create subsections
	
	
	   \textbf{Search feature} :\begin{itemize}\item  The search feature should be easy for the user to find
              \item Results from any search should be easy to use
                     \end{itemize}
    \textbf{GPS} : 
    \begin{itemize}\item The results displayed in the map view should be user friendly and easy to understand
           \item Selecting a pin on the map should only take one click
           \item Retreving customer's location should'nt take too long
           \item Getting the nearest restuarant also shound'nt take too long
           \item Preferable timmings
                %$->$\textbf{METER}: Measurements obtained from 1000 searches during testing.\\        
                %$->$ \textbf{MUST}: No more than 15 seconds 100$%$ of the time.
	            %\\$->$ \textbf{WISH}: No more than 10 second 100$%$ of the time.		
    
   \end{itemize}
   \textbf{others} : \begin{itemize}\item Dropdown menus should be identified and used easily
             \item Submitting an order should take as lil time as possible
             \item Admin (restaurant) should recieve an order as soon as the delay time had elapsed
             \item If the system loses the connection to the Internet or to the GPS device or the system gets some strange input, the user should be informed
	   \end{itemize}
	\end{homeworkSection}
 \begin{homeworkSection}{Safety and Security Requirements}

   \begin{itemize}\item Communication between the system and server should be secured. Messages for log-in communication should be encrypted so that 
     other would not get any infomation from them.

   \item Admin Account should be secured,the Resturant owner should not be allowed to log in (for a certain period of time) after three times of failed log-in attempts. And this must be reported to the developers to check if the system is being attacked or it was just a mistake by the restaurant owner.

    \item Since the system require users to pay before an order is made, Customers' sensative information should be heavily protected.
    
    \item Orders should'nt be mixed up,make sure that the order made goes to the correct restaurant, so to avoid this orders may be stored with 
     respect to restuarants. If a user chooses to sign up, their email adress/cellphone no should be protected.
   \end{itemize}
 \end{homeworkSection}
 
 \begin{homeworkSection}{Software Quality Attributes}
    \textbf{ Maintainability }: The app shall be written in a way that allows scalability of database resources and maintainability of User information.

    \textbf{Availability} : The server shall be running persistent connections to restaurants that are open and are using their Client -side app.

    \textbf{Reliability} : Users will always be shown all restaurants in 20km radius that are registered with this system.

 \end{homeworkSection}
	
\end{homeworkProblem}


%----------------------------------------------------------------------------------------

\end{document}