%%%%%%%%%%%%%%%%%%%%%%%%%%%%%%%%%%%%%%%%%
% Structured General Purpose Assignment
% LaTeX Template
%
% This template has been downloaded from:
% http://www.latextemplates.com
%
% Original author:
% Ted Pavlic (http://www.tedpavlic.com)
%
% Note:
% The \lipsum[#] commands throughout this template generate dummy text
% to fill the template out. These commands should all be removed when 
% writing assignment content.
%
%%%%%%%%%%%%%%%%%%%%%%%%%%%%%%%%%%%%%%%%%

%----------------------------------------------------------------------------------------
%	PACKAGES AND OTHER DOCUMENT CONFIGURATIONS
%----------------------------------------------------------------------------------------

\documentclass{article}
\usepackage{forest}

\usepackage[T1]{fontenc}
\usepackage[utf8x]{inputenc}
\usepackage[french]{babel}

\usepackage{tikz}

\sloppy
\hyphenpenalty 10000000

\usepackage{fancyhdr} % Required for custom headers
\usepackage{lastpage} % Required to determine the last page for the footer
\usepackage{extramarks} % Required for headers and footers
\usepackage{graphicx} % Required to insert images
\usepackage{lipsum} % Used for inserting dummy 'Lorem ipsum' text into the template
\usepackage[table]{xcolor}
\newcommand\tab[1][1.7cm]{\hspace*{#1}}
% Margins
\topmargin=-0.45in
\evensidemargin=0in
\oddsidemargin=0in
\textwidth=6.5in
\textheight=9.0in
\headsep=0.25in 

\linespread{1.1} % Line spacing

% Set up the header and footer
\pagestyle{fancy}
\lhead{\hmwkAuthorName} % Top left header
\chead{\hmwkClass} % Top center header
\rhead{\firstxmark} % Top right header
\lfoot{\lastxmark} % Bottom left footer
\cfoot{} % Bottom center footer
\rfoot{Page\ \thepage\ of\ \pageref{LastPage}} % Bottom right footer
\renewcommand\headrulewidth{0.4pt} % Size of the header rule
\renewcommand\footrulewidth{0.4pt} % Size of the footer rule

\setlength\parindent{0pt} % Removes all indentation from paragraphs

%----------------------------------------------------------------------------------------
%	DOCUMENT STRUCTURE COMMANDS
%	Skip this unless you know what you're doing
%----------------------------------------------------------------------------------------

% Header and footer for when a page split occurs within a problem environment
\newcommand{\enterProblemHeader}[1]{
\nobreak\extramarks{#1}{#1 continued on next page\ldots}\nobreak
\nobreak\extramarks{#1 (continued)}{#1 continued on next page\ldots}\nobreak
}

% Header and footer for when a page split occurs between problem environments
\newcommand{\exitProblemHeader}[1]{
\nobreak\extramarks{#1 (continued)}{#1 continued on next page\ldots}\nobreak
\nobreak\extramarks{#1}{}\nobreak
}

\setcounter{secnumdepth}{0} % Removes default section numbers
\newcounter{homeworkProblemCounter} % Creates a counter to keep track of the number of problems

\newcommand{\homeworkProblemName}{}
\newenvironment{homeworkProblem}[1][Problem \arabic{homeworkProblemCounter}]{ % Makes a new environment called homeworkProblem which takes 1 argument (custom name) but the default is "Problem #"
\stepcounter{homeworkProblemCounter} % Increase counter for number of problems
\renewcommand{\homeworkProblemName}{#1} % Assign \homeworkProblemName the name of the problem
\section{\homeworkProblemName} % Make a section in the document with the custom problem count
\enterProblemHeader{\homeworkProblemName} % Header and footer within the environment
}{
\exitProblemHeader{\homeworkProblemName} % Header and footer after the environment
}

\newcommand{\problemAnswer}[1]{ % Defines the problem answer command with the content as the only argument
\noindent\framebox[\columnwidth][c]{\begin{minipage}{0.98\columnwidth}#1\end{minipage}} % Makes the box around the problem answer and puts the content inside
}

\newcommand{\homeworkSectionName}{}
\newenvironment{homeworkSection}[1]{ % New environment for sections within homework problems, takes 1 argument - the name of the section
\renewcommand{\homeworkSectionName}{#1} % Assign \homeworkSectionName to the name of the section from the environment argument
\subsection{\homeworkSectionName} % Make a subsection with the custom name of the subsection
\enterProblemHeader{\homeworkProblemName\ [\homeworkSectionName]} % Header and footer within the environment
}{
\enterProblemHeader{\homeworkProblemName} % Header and footer after the environment
}
   
%----------------------------------------------------------------------------------------
%	NAME AND CLASS SECTION
%----------------------------------------------------------------------------------------

\newcommand{\hmwkTitle}{Fast-Food Management System Final Report} % Assignment title
\newcommand{\hmwkDueDate}{Monday,\ October\ 02\ 2017} % Due date
\newcommand{\hmwkClass}{Software Engineering} % Course/class
\newcommand{\hmwkClassTime}{COMS3002} % Class/lecture time
\newcommand{\hmwkClassInstructor}{Daniel Holmes} % Teacher/lecturer
\newcommand{\hmwkAuthorName}{Group 4} % Your name
\newcommand{\hmwGroupMembers}{Front-end: Kamogelo Khamiwa(674815) & Chuba Moganedi(716705)}
\newcommand{\hmkGroupMembers}{Back-end: Nthikeng Letsoalo(1037720) & Mohale Nakana(1174093)}
%----------------------------------------------------------------------------------------
%	TITLE PAGE
%----------------------------------------------------------------------------------------

\title{
\vspace{2in}
\textmd{\textbf{\hmwkClass:\ \hmwkTitle}}\\
\normalsize\vspace{0.1in}\small{Due\ on\ \hmwkDueDate}\\
\vspace{0.1in}\large{\textit{\hmwkClassInstructor\ \hmwkClassTime}}\\
\vspace{3in}
\vspace{0.1in}\large{{\hmwGroupMembers}}\\
\vspace{0.1in}\large{{\hmkGroupMembers}}
}

\author{\textbf{\hmwkAuthorName}}

\date{} % Insert date here if you want it to appear below your name

%----------------------------------------------------------------------------------------

\begin{document}
\maketitle

%----------------------------------------------------------------------------------------
%	TABLE OF CONTENTS
%----------------------------------------------------------------------------------------

\setcounter{tocdepth}{1} % Uncomment this line if you don't want subsections listed in the ToC

\newpage
\tableofcontents
\newpage

%----------------------------------------------------------------------------------------
%	PROBLEM 1
%----------------------------------------------------------------------------------------

% To have just one problem per page, simply put a \clearpage after each problem



%----------------------------------------------------------------------------------------
%	PROBLEM 2
%----------------------------------------------------------------------------------------


\begin{homeworkProblem}[Introduction]
	\begin{homeworkSection} { - Problem Statement} 
	
The ordering protocol used by fast food restaurants should aim on getting the meals of a customer being prepared within a preferred time,restaurants should avoid long queues on their tills so that the customers won't have to waste their time queuing to make orders.

Unfortunately the current ordering system used is not efficient and we see the restaurants loosing customers(because no one want to join a long queue on an empty stomach).The system also course financial burdens on the restaurants since they have hire extra stuff in order to try serve the long queues fast.

Restaurants can use the Fast Food Management system implemented by Group 4 to over come this challenge(below is the description of this Fast Food Management System).	
		
	\end{homeworkSection}	
     
     \begin{homeworkSection} { - Overview of the project} 
      
We are going to build a web application to facilitate making orders at selected Fast-Food restaurants.
Our application will allow users to sign-up an account, select a restaurant within a certain radius of their
geo-location and upon selecting their order they may choose to eat-in at the restaurant or pick up their
order. An order is consider complete after payment has been settled using a unique reference number
issued when the order was placed. We will ensure that all payments are secured and customer credentials
protected so that no fraudulent activities may occur.
      
     \end{homeworkSection}     
     
     \begin{homeworkSection} { - Project Objectives} 
       
        \begin{itemize}
 		 
 		 \item Enable users to create an account using their email address to log their activity.
\item Login page for restaurants customers and admins
\item Store payment credentials together with the user account

\item Make online payment secure

\item Login accomplished by entering username(email) and password.

\item Use Geo-location to isolate near-by Fast-food restaurants.

\item Restaurant menu to be displayed in-app 

\item Create easy to navigate menus

\item A shopping basket implementation to store selected items.

\item Order receipt to be generated and sent to customer for book-keeping.
  		
		\end{itemize}    
       
     \end{homeworkSection}
     
     
     \begin{homeworkSection} { - Stakeholders} 
      
       
        \begin{itemize}
            \item \textbf{Developers}:
            
                \begin{itemize}
                    \item \textbf{Front-end}:
                    Front-end developers will be using Android Studio and Java to create the aesthetics of the application
                    \item\textbf{Back-end}:
                    Back-end developers will be using the online Firebase database Console to maintain data integrity throughout the system for our Customers and Clients.
                \end{itemize}
        
            \item \textbf{The Client}:
            Our clientele will be Fast-Fast restaurant industry. We provide a restaurant with the means to optimize their service delivery by providing a means to expand their customer base. Not only will residents of a particular location be potential customers but now anyone with the application could be a customer.
            
            \item \textbf{Hands-on users of the product}:
            Our customers are the fast-food consumers who need efficient preparation and delivery of placed orders.
            
                \textbf{Admin} :
                \begin{itemize}
                  \item Admin can view orders made by customers
                  \item Can change menu of their restaurant 
                  \item They can also give feed back
                \end{itemize}
                    \textbf{ Customers} :
                \begin{itemize}
                 \item Customers can create an account
                 \item Customers can make an order, they can cancel an order (before the delay time have not elapsed)
                 \item They can give feedback     
                 \item They can also specify a restaurant they want to place on order to
                 \end{itemize} 
         
         The only training necessary is for admin users who will need basic computer literacy.
            \item \textbf{Priorities assigned to users}:
            
                \textbf{ Customers} : Have the least privileged permissions. Customers need only interact with the database passively when signing up and Logging in to the application, their priority is to order food and receive a reference number for order collection.
                
                \textbf{ Admin} :Have the most privileged permissions. Admin reserves the right to cancel an order on non-collection and may change their menu selections. Admin priorities are limited to updating the menu selections, acknowledging incoming orders and preparing the customers order.
        \end{itemize}
    
    \end{homeworkSection}
    
    \begin{homeworkSection}{Mandated Constraints}
    
    This section describes constraints on the eventual design of the product. They are the same as other requirements except that constraints were mandated at the beginning of the project.
        \begin{itemize}
            \item Pick-up location must be selected by the client.
            \item There must be a time period allowed for changes to a customers order.
            \item The service is only usable for customers that issue orders within 20km of a pick-up or eat-in restaurant.
        \end{itemize}
    
    \end{homeworkSection}
     
\end{homeworkProblem}


$*************************************************************************************$

\cleardoublepage

\begin{homeworkProblem}[SOFTWARE REQUIREMENT SPECIFICATION]
	
    \begin{homeworkProblem}[Introduction]
	\begin{homeworkSection}{Purpose} %/ use this block to create subsections
	The purpose of this document is to provide a detailed description of the final Fast Food Management System we intend to implement with SCRUM development life cycle.
	
	\end{homeworkSection}
	\begin{homeworkSection}{Intended Audience and Reading Suggestions}
	
	Software development teams, our peers and superiors alike are who this document is addressed to. Now, this document is written in an effort to facilitate cross-departmental contribution to this project. We open with an overall description to state our intended end product followed by external interface requirements to expand upon our development strategies. System features are included to further modularize the necessary tasks our project must undertake. Nonfunctional requirements explain our intended performance security goals. 
	\end{homeworkSection}
	
	\begin{homeworkSection}{Product Scope}
	
	\textbf{Current Situation}:
	
	Similar services are being provided by Square Order, Grub-Hub and eHungry.
	These applications provide users the means to place orders online through a
	marketspace site implementation that is a platform on which restaurants can
	host their mobile ordering functionality. We intend on creating an application
	that maintains the individualism of each restaurant. Our application is
	customer oriented in that instead of pooling all our registered restaurants we
	will present the most convenient establishments based on the customers
	geo-location. 
	\end{homeworkSection}
	
	\begin{homeworkSection}{Definitions}
	\textbf{Three-tier client-server architecture} : System Architecture that will define three logically independent tiers, namely, Presentation, Data management and Domain logic.
	
\textbf{Geo-location} : An estimation of real world geographical location.

\textbf{Service providers} : Refers to restaurants and other registered fast food outlets

\textbf{PostgreSQL} : An open source Database management system.

\textbf{iOS(X)} : Operating system used by mobile Apple devices.

\textbf{Android} : Google's Most common open-source operating system used by mobile devices.

\textbf{Windows 10 mobile} : Windows operating system ported for mobile device functionality.

\textbf{Firebase} : A mobile and web application development platform.

\textbf{JSON} : (JavaScript Object Notation) Is used to transmit data between server and web application.

\textbf{JQuery} : JavaScript framework that allows manipulation of a web page to enble user interaction.

	\end{homeworkSection}
	\begin{homeworkSection}{References}
	
	\textbf{- van Vliet, H, (2007), Software Engineering Principles and Practice,Wiley.}
	
	
	\end{homeworkSection}
	
\end{homeworkProblem}

%----------------------------------------------------------------------------------------
%	PROBLEM 2
%----------------------------------------------------------------------------------------

\begin{homeworkProblem}[Overall Description]
	\begin{homeworkSection}{Product perspective} %/ use this block to create subsections
	 The application will take the form of a three-tier client-server architecture. On accessing the application, customers are able to place orders at their selected establishment. Geo-location information is gathered and fast food restaurants will appear in order of proximity from our customers recorded geo-location. Menu presentation of the selected restaurant is followed by a place order interface the customer will use to select their desired items. Payment credentials will be secured and stored prior to issuing an order number to our customer which must be presented on collection of their order. Administrator functionality will include the option to alter/delete menu presentations with store and review functionality for customer orders, this is necessary for monthly reports.
	
	% one last paragraph about security.
	\end{homeworkSection}
	
	\begin{homeworkSection}{Product functions} %/ use this block to create subsections
		\begin{itemize}
 		 \item Enable the user to log on to the system so that orders can be tracked
  		\item Make use of geo-location services to give the user recommendations on places to eat within a 20km radius
  		\item Have a user friendly menu to avoid confusion when making orders
  		\item When an order has been made, give a the user a period of 2 minutes to make changes to the order
  		\item Make online payment possible via credit/debit card
  		\item Make the online payment secure
  		\item Make use of geo-location services to give user directions to the restaurant 
  		\item Keep track of all the orders coming through using  numbers 
  		
		\end{itemize}
	\end{homeworkSection}

	\begin{homeworkSection}{User Classes and Characteristics} %/ use this block to create 		subsections
	From the consumer side, anyone with access to a smartphone and a working bank account is a potential user of the software. To narrow it down a bit, it will be well suited for the working class, as they have long working hours and they are often stuck in traffic. Thus this software will be perfect for them, i.e. they can order whilst they are stuck in traffic, and by the time they get to the food outlet their order will be ready to take home. 
	
	Looking at the service providers (restaurants and fast food outlets), the system will be used by trained staff members. Quick training is to be received upon installation of the software.
	\end{homeworkSection}

	
	\begin{homeworkSection}{Operating Environment} %/ use this block to create subsections
	This software is designed to run on android smartphones. It shall be developed using Android Studio IDE in java, in conjunction with the Google API to provide for geographic-location services. Firebase will be the open source database management system that will be in place to implement all relationships that exist between datasets.
	
	\end{homeworkSection}
	
	\begin{homeworkSection}{Design and Implementation Constraints} %/ use this block to create subsections
	High accuracy GPS is very crucial for the system to work at it's best. Also the app should not use more than 128MB of the RAM and take minimum CPU time as that may hamper smooth operation of the entire smartphone.
	\end{homeworkSection}
	
	
	\begin{homeworkSection}{User Documentation} %/ use this block to create subsections
	
	\begin{itemize}
	\item Consumers will be given a 1-page comprehensive graphical user-manual
	\item Services providers will also be given their own user-manual which will also double as training manual
	\end{itemize}
	\end{homeworkSection}

\end{homeworkProblem}



\begin{homeworkProblem}[External Interface Requirements]
		\begin{homeworkSection}{Customer Interfaces} %/ copy this block to create subsections
	Currently we are only focusing on the mobile versions, Android mobile in particular. That being said, the main feel of the app will be a modern design so tiles and buttons will be constructed using XML.
	
	The first page is a Home screen. The footer of the app will have 3 fixed buttons: 
		\begin{itemize}
		\item Order - To show pending orders and well as approved orders
		\item Logout - To go to the Home screen
		\item (Idle) - This button on the footer doesn't have a function as of yet
		\end{itemize}
	\end{homeworkSection}
	\begin{homeworkSection}{Hardware Interfaces} %/ copy this block to create subsections
	The app should run on any Android device running API 21 or later.
	\end{homeworkSection}
	\begin{homeworkSection}{Software Interfaces} %/ copy this block to create subsections
		A noSQL database management system,Firebase shall be used for this three-tier system. The Google Maps API should be used for users to visualize their locations as well as locations of the local restaurants.
	\end{homeworkSection}
	\begin{homeworkSection}{Communication Interfaces} %/ copy this block to create subsections
	HTTP will be our communication protocol. We expect encrypted data to shared as JSON arrays. We want to minimize response time so persistent open threads will need to be available on the server-side.
	\end{homeworkSection}

\end{homeworkProblem}


\begin{homeworkProblem}[Functional Requirements: System Features]
	\begin{homeworkSection}{Pages and permissions} %/ copy this block to create subsections
	\begin{itemize}	
	\item Restaurant locater
	\end{itemize}

 	- Identify restaurant name
 
 	- Calculate distance to restaurant

 	- Connect to restaurant server

 	- Load admin/customer interface

	\begin{itemize}
	\item Admin
	\end{itemize}

 	- Can view and change the restaurant menu 
 	
 	- Order coordination to ensure efficient meal preparation for shortest waiting times
 	
 	- Can view user accounts and see accumulated receipts and log

	\begin{itemize}
	\item Customer 
	\end{itemize}

	- Places an order online

	- Selects preferred restaurant and places order

	- Payment credentials are stored and protected

	- Upon successful placement of order, a unique reference number is given to the customer. The number shall be used to access the customers payment credentials

	- Customer selects preparation method, eat-in/collect. Their order is then verified and at this point the customer may change their order

	- Order is verified and stored in archive. Approximate preparation time and map route(address) is displayed
	
	\begin{itemize}
	\item Customer locator
	\end{itemize}
	
	- Identify customer's location

	- Retrieve proximity information from restaurant server and display restaurant location to customer and route.

	\end{homeworkSection}

	\begin{homeworkSection}{Logistic management}

	\begin{itemize}
	\item User accounts
	\end{itemize}

	- Users can register an account that will need their email and password in this way customers can store their receipts with us

	\begin{itemize}
	\item Menu display
	\end{itemize}

	- Restaurant menus need to be quite presantable

	- The order placing interface will comprise of drop box item selections and view of accumulated cost
	
	\begin{itemize}
	\item Unique reference number
	\end{itemize}
	
	\end{homeworkSection}
	
	

\end{homeworkProblem}


\begin{homeworkProblem}[Non-functional Requirements]
		
	\begin{homeworkSection}{Performance Requirements} %/ use this block to create subsections
	
	
	   \textbf{Search feature} :\begin{itemize}\item  The search feature should be easy for the user to find
              \item Results from any search should be easy to use
                     \end{itemize}
    \textbf{GPS} : 
    \begin{itemize}\item The results displayed in the map view should be user friendly and easy to understand
           \item Selecting a pin on the map should only take one click
           \item Retrieving customer's location shouldn't take too long
           \item Getting the nearest restaurant also shouldn't take too long
           \item Preferable timmings
                %$->$\textbf{METER}: Measurements obtained from 1000 searches during testing.\\        
                %$->$ \textbf{MUST}: No more than 15 seconds 100$%$ of the time.
	            %\\$->$ \textbf{WISH}: No more than 10 second 100$%$ of the time.		
    
   \end{itemize}
   \textbf{others} : \begin{itemize}\item Dropdown menus should be identified and used easily
             \item Submitting an order should take as little time as possible
             \item Admin (restaurant) should receive an order as soon as the delay time had elapsed
             \item If the system loses the connection to the Internet or to the GPS device or the system gets some strange input, the user should be informed
	   \end{itemize}
	\end{homeworkSection}
 \begin{homeworkSection}{Safety and Security Requirements}

   \begin{itemize}\item Communication between the system and server should be secured. Messages for log-in communication should be encrypted so that 
     other would not get any information from them.

   \item Admin Account should be secured,the Restaurant owner should not be allowed to log in (for a certain period of time) after three times of failed log-in attempts. And this must be reported to the developers to check if the system is being attacked or it was just a mistake by the restaurant owner.

    \item Since the system require users to pay before an order is made, Customers' sensitive information should be heavily protected.
    
    \item Orders shouldn't be mixed up,make sure that the order made goes to the correct restaurant, so to avoid this orders may be stored with 
     respect to restaurants. If a user chooses to sign up, their email address/cellphone no should be protected.
   \end{itemize}
 \end{homeworkSection}
 
 \begin{homeworkSection}{Software Quality Attributes}
    \textbf{ Maintainability }: The code should be written in a way that allows extension of functions or implementation of new function, for upgrading to be easy.
    
   \textbf{ Portability and flexibility} : The system should be accessible from any well known browser (e.g. Google Chrome, Mozilla) other than the app.

    \textbf{Availability} : The web should be reachable whenever it is needed, and anyone in South Africa should be able to access it (so long as they are connected).

    \textbf{Reliability} : Users should get correct results from any search they make.

 \end{homeworkSection}
 
 \begin{homeworkSection}{Business Rules} 
      
       \textbf{Admin} :\begin{itemize}
       
  \item      Admin can view orders made by customers
  \item   Can change menu of their restaurant 
  \item    They can also give feed back
 \end{itemize}
   \textbf{ Customers} :\begin{itemize}
     \item Customers can create an account
     \item Customers can make an order, they can cancel an order (before the delay time have not elapsed)
     \item They can give feedback     
     \item They can also specify a restaurant they want to place on order to
    \end{itemize}       
      
 \end{homeworkSection}
	
\end{homeworkProblem}

	
	
\end{homeworkProblem}

\begin{homeworkProblem}[ PART I : Front-end Document ]

\begin{homeworkProblem}[PROJECT DESIGN DOCUMENT]
	
\end{homeworkProblem}

   \begin{homeworkProblem}[Product Backlog]
   
    \begin{enumerate}
      \item login page for restaurant's, customers and administration
      \item An option to order without logging in 
      \item Location finder for restaurants within a radius of 20km
      \item Display the menu for the selected restaurants
      \item Checkbox menu for selecting meals
      \item Page for making payments
      \item Indication of the distance between the customer and the restaurants
      \item An option to search for a restaurant
     
    \end{enumerate}    
   \end{homeworkProblem}
   
$***************************************************************************************$   
   
   
\begin{homeworkProblem}[SPRINT PLANNING AND SPRINT RETROSPECTIVES]



\begin{tabular}{ |p{3cm}||p{4cm}|p{2cm}|p{4cm}|  }
 \hline
 \multicolumn{4}{|c|}{\textbf{Sprint 1}} \\
 \hline
 \textbf{Item no} & \textbf{Description} & \textbf{Estimated \hspace{0.5cm} hours} & \hspace{0.5cm}\textbf{ By}\\
 \hline
 01  & Login page for customers and admins &  &   Kamogelo,Garfield\\
  \hline
 02 &  An option to order without logging in  &   & Kamogelo,Garfield\\
  \hline
 03 & Location finder for for restaurants within a radius of 20km with reference to the customer & &  Kamogelo,Garfield \\
  \hline
 04    & Display the menu for the selected restaurant & & Kamogelo,Garfield\\
 \hline
 05  & an Option for selecting meals &  &   Kamogelo,Garfield\\
  \hline
 06  & Page for making payments &  &  Kamogelo,Garfield\\
  \hline

\end{tabular}

\begin{homeworkSection} {Working Schedule for Sprint 1}
     
       Below is the our schedule of working on the first sprint : \\ 
       
       \begin{tabular}{ |p{3cm}||p{3cm}|p{5cm}|p{3.5cm}|  }
 \hline
 \multicolumn{4}{|c|}{\textbf {Sprint 1 Work Schedule}} \\
 \hline
 \textbf{Date} & \textbf{Day} & \textbf{Duties} & \hspace{0.5cm} \textbf{By}\\
 \hline
 21 Aug  & Monday & Sprint planning &   Kamogelo,Garfield\\
  \hline
23 Aug  & Wednesday & Login page for restaurant customers and admins &   Kamogelo,Garfield\\
  \hline
  25 Aug  & Wednesday & An option to make orders without logging in &   Kamogelo,Garfield\\
  \hline
 28 Aug  & Friday & Location finder for restaurants within a radius of 20km &   Kamogelo,Garfield\\
  \hline
 30 Aug  & Monday & Display the menu for the selected restaurant to the customer &   Kamogelo,Garfield\\
  \hline
 01 Sept  & Wednesday & an Option for selecting meals &   Kamogelo,Garfield\\
  \hline
 04 Sept  & Friday & Payment page &   Kamogelo,Garfield\\
  \hline

\end{tabular}      
       
     \end{homeworkSection}
     
\begin{homeworkProblem}[Sprint reviews]

    \textbf{Sprint 1 day 0(21 Aug) :} This is the first day of our sprint 1 we plan the project as a whole,we all gathered and list all the responsibilities and requirements from both front-end and back end.We also reached an agreement to meet every two days(Weekends excluded).
    After we met as front-ends and we did our sprint 1 planning.
    
   \textbf{Sprint 1 day 2(23 Aug)  :}  
      \begin{itemize}  
      
        \item Status : Sprint 1 planning was complete
        \item Challenges : we had no challenges
        \item Solution   : we move to the next task since we faced no challenges.
        \item To-do : Login page for restaurants' customers and admins
      
      \end{itemize}
      
    \textbf{Sprint 1 day 3(24 Aug)  :} 
      \begin{itemize}  
      
        \item  Status : Still busy login page.
        \item Challenges : The app was not eye-catching.
        \item Solution   : Improve the layout of the app.
        \item To-do : Improve the layout of the app.
      
      \end{itemize}      
     
     \textbf{Sprint 1 day 4(25 Aug) :} 
      \begin{itemize}  
      
        \item  Status : Done with the login.
        \item Challenges : The app was not very eye-catching.
        \item Solution   : We reserved it for later.It was way too better to be accepted and users could see everything \tab clear we move to the next item.
        \item To-do : Add an option to order without logging in.  
      
      \end{itemize}
      
       \textbf{Sprint 1 day 7(28 Aug):}  
      \begin{itemize}  
      
        \item  Status : Done adding an option to order without logging in.
        \item Challenges : none.
        \item Solution   : none.
        \item To-do : Location finder for restaurants within a radius of 20km.
      
      \end{itemize}
      
       \textbf{Sprint 1 day 8(29 Aug) :} 
      \begin{itemize}  
      
        \item  Status : Still busy with location finder.
        \item Challenges : Doesn't show restaurants only.
        \item Solution   : Look for third parties to help in displaying restaurants only.
        \item To-do : Look for third parties to help in displaying restaurants only.
      
      \end{itemize}
      
      \textbf{Sprint 1 day 9(30 Aug)  :} 
      \begin{itemize}  
      
        \item  Status : Location finder On-hold.
        \item Challenges : Still doesn't show restaurants only.
        \item Solution   : We put it on hold and we will continue with it later.
        \item To-do : Display the menu for the selected restaurant to the customer.
      
      \end{itemize}    
      
       \textbf{Sprint 1 day 11(01 Sept)  :} 
      \begin{itemize}  
      
        \item  Status : Done with displaying the menu.
        \item Challenges : Doesn't display a specific menu for a specific restaurant.
        \item Solution   : We use a one menu for all restaurants(a general menu).
        \item To-do : Option for selecting meals.
      
      \end{itemize}     
      
      \textbf{Sprint 1 day 11(04 Sept)  :} 
      \begin{itemize}  
      
        \item  Status : Done with the option for selecting meal(Used checkbox).
        \item Challenges : none.
        \item Solution   : none.
        \item To-do : Page for making payments.
        
        
      
      \end{itemize}  
      
      \textbf{Sprint 1 day 12(06 Sept)  :} 
      \begin{itemize}  
      
        \item  Status : Done with the payments page(Used checkbox).
        \item Challenges : Couldn't use real accounts.
        \item Solution   : we used a fake one.
        \item To-do : Attend the final meeting.  
      \end{itemize} 
      
     	We were almost done but ...................  
       
      
\end{homeworkProblem}
    	
	
\end{homeworkProblem}

\begin{homeworkProblem}[SPRINT RETROSPECTIVE]
	
\end{homeworkProblem}


\begin{homeworkProblem}[DESCRIPTION OF RELEVANT MODULES]

\begin{itemize}
    \item \textbf{Decomposition Description}
    
    Our application Fast-Food management system is comprised of 9 separate modules namely:
            \begin{itemize}
                \item \textbf{MainActivity}
                \item \textbf{SignUpActivity}
                \item \textbf{SignInActivity}
                \item \textbf{AccountActivity}
                \item \textbf{PaymentActivity}
                \item \textbf{MapActivity}
                \item \textbf{Dataparser}
                \item \textbf{DownloadUrl}
                \item \textbf{GetNearbyPlaceData}
            \end{itemize}
            
    
    \item \textbf{Dependencies Description}
    
    The following tree describes the manner in which certain modules are dependent on the output of certain other modules.
    
        \begin{forest}
          [MainActivity, inner sep=1pt
            [SignUpAct 
             [AccountAct
                [PaymentAct
                    [MapActivity]
                    ]
                ]
            ]
            [SignInAct
              [AccountAct
                [PaymentAct
                    [MapActivity]
                    ]
                ]
            ]
          ]
        \end{forest}
        
    \item \textbf{Interface Description}
    
        \begin{itemize}
            \item \textbf{MainActivity:}
            
            -Presents the user with the option to register an account with our database. The user whom is already a registered user may use this page to Sign-In to their Account.
            
            -Presents the user with the option to access the site as a Fast-Food client to register an account with us.
            
            -Finally it provides seldom users of the application the option to order food without registering an account first.
            
            \item \textbf{SignUpActivity:}
            
            -This is the module is used by new users to register an account
            
            \item \textbf{SignInActivity:}
            
            -Presents registered users with a logging in page to access their respective accounts.
            
            \item \textbf{AccountActivity:}
            
            - This module provides both registered and unregistered users the interface necessary to select a preferred restaurant and place an order for food.
            
            \item \textbf{PaymentActivity:}
            
            -Users are lead to this page after having selected a restaurant and selected the relevant menu items. They are prompted to enter the payment credentials to complete the placing of an order.
            
            \item \textbf{MapActivity:}
            
            -This module is responsible for the displaying of the map that is generated after a customers geo-location information is gathered. It is called again after an order has been finalized to display the route to the selected restaurant.
            
            \item \textbf{Dataparser:}
            
            -This module receives JSON object data from the network and organises the infomation into hashmaps.
            
            \item \textbf{DownloadUrl:}
            
            -This is used for setting up the connection between a customer and google to recieve a URL that will be used by GetNearByPlace to detail the map displayed by MapActivity.
            
            \item \textbf{GetNearByPlaces:}
            
            -Is used to augment the map display to show the relevant locations on the map that fall within or radius of action.
        \end{itemize}
    \item \textbf{Detail Description:}

	\begin{itemize}
	\item \textbf{MainActivity}:
	    \begin{itemize}
	        \item INPUT:
	        
	        This method is called on initialization.
	        \item OUTPUT:
	        
	        Provides user with the interaction options such as SignIn and SignUp. 
	    \end{itemize}
	    
	\item \textbf{SignUpActivity}: 
	    \begin{itemize}
	        \item INPUT:
	        
	        Customer enters their email address and password which will be stored
	        in the user database.
	        \item OUTPUT:
	        
	        Once successfully stored in our database the customer will now have an
	        active account through which they may access the application
	        functionality.
	    \end{itemize}
	    
	\item \textbf{SignInActivity}: 
	    \begin{itemize}
	        \item INPUT:
	        
	        Customer enters their email address and password which will be checked
	        in the user database as to whether the user exists.
	        \item OUTPUT:
	        
	        Once successfully validated in our database the customer will now be
	        taken to their Home page where they may choose to log-out or initiate the restaurant locator and proceed to order food.
	    \end{itemize}
	    
	 \item \textbf{MapActivity}: 
	    \begin{itemize}
	        \item INPUT:
	        
	        Geo-location of the customer's device location is gathered and
	        transmitted to our server.
	        \item OUTPUT:
	        
	        Based on this information a relevant map of the customer's location is displayed.
	    \end{itemize}
	    
	 \item\textbf{GetNearbyPlaceData}:
	        \begin{itemize}
	            \item INPUT:
	            
	            initialized by MapActivity.
	            \item OUTPUT:
	            
	            will provide the user with a list of near-by restaurants.
	        \end{itemize}
	        
	\item \textbf{DownloadUrl}:
	        \begin{itemize}
	            \item INPUT:
	            
	            Initialized by GetNearByPlaceData.
	            \item OUTPUT:
	            
	            Google URL to GoogleMaps API.
	        \end{itemize}
	        
	\item \textbf{Dataparser}:
	       \begin{itemize}
	           \item INPUT:
	           
	           Initialized by GetNearByPlaceData.
	           \item OUTPUT:
	           
	           It does not output rather is organizes the network data in hash-tables.
	       \end{itemize}
	    
	\item \textbf{AccountActivity}: 
	    \begin{itemize}
	        \item INPUT:
	        
	        Customer enters their menu selections and proceed to place their order.
	        The order is received by our back-end and payment must be made before
	        finalizing the order.
	        \item OUTPUT:
	        
	        Once an order is successfully placed the customer is issued a unique
	        reference number which will be required by the establishment to collect
	        their order and complete the transaction.
	    \end{itemize}  
	    
	\item \textbf{PaymentActivity}:
	\begin{itemize}
	     \item INPUT:
	       
	       Customer's payment credentials used to finalize the booking of an order.
	        \item OUTPUT:
	       
	       The location of the restaurant which the order was placed.
	\end{itemize}
	       
	
	\end{itemize}
\end{itemize}
	
\end{homeworkProblem}


%----------------------------------------------------------------------------------------

\end{homeworkProblem} %//END OF PART I



\begin{homeworkProblem}[ PART II : Back-end Document ]

    \textbf{Implementation modules}
    
    \begin{itemize}
        \item \textbf{AccountActivity:}
            \begin{itemize}
                \item public void placeOrder()
                
                This module takes in user input and delivers the information to the back-end.
                
                \item public boolean onCreatOptionsMenu()
                
                It creates an interactive toolbar with menu items
                
                \item public boolean onOptionsItemSelected()
                
                It is implemented as an event listener to enable logout functionality on the toolbar.
                
                \item private void logout()
                
                When called it ends a users active session.
            \end{itemize}
        \item \textbf{Dataparser:}
                \begin{itemize}
                    \item private HashMap<String,String>getPlace()
                    \item private List<String,String>getPlaces()
                    \item public List<HashMap<String,String>>Parse()
                \end{itemize}
        \item \textbf{DownloadUrl:}
                \begin{itemize}
                    \item public String readUrl()
                    
                    Reads the incoming URL.
                \end{itemize}
        \item \textbf{GetNearByPlacesData:}
                \begin{itemize}
                    \item protected String doInBackgroud()
                    \item protected void onPostExecute()
                    \item private void showNearByPlaces()
                \end{itemize}
        \item \textbf{MainActivity:}
                \begin{itemize}
                    \item public void onAuthStateChanged()
                    
                    This module performs a check to determine whether a user is already logged in to the database.
                    
                    \item protected void onActivityResult()
                    
                    It returns the address that the customer has selected from the  map.
                    
                    \item protected void onStart()
                    
                    This method adds an event listener for the Auth state*
                    
                \end{itemize}
        \item \textbf{MapActivity:}
                \begin{itemize}
                    \item public void onRequestPermissionsResult()
                    \item public void onMapReady()
                    \item private String getUrl()
                    \item public boolean cheakLocationPermission()
                    \item public void onConnectionSuspended()
                    \item public void onConnectionFailed()
                    \item public void onLocationChanged()
                \end{itemize}
        \item \textbf{PaymentActivity:}
                \begin{itemize}
                    \item public void onClick()
                \end{itemize}
        \item \textbf{SignInActivity:}
                \begin{itemize}
                    \item public void onAuthStateChanged()
                    	Checking whether the  user as previously sign in, if he is still signed in it opens the Google place picker to let the user start selecting the restaurant of his choice;
                    \item protected void onActivityResult()
                    	Method is called when the google place picker returns after the user has selected the restaurant of his choice, we use this to get the name and address of the place.
                    \item protected void onStart()
                    	An android method the is called when the Activity has started up, inside this method we put the event listener for the firebase Authentication services.
                    \item private void startSignIn()
                    	Method that enables the user to log in to the app.
                    \item public void onComplete()
                    	Method to check if the user has successfully signed into the application.
                \end{itemize}
        \item \textbf{SignUpActivity:}
                \begin{itemize}
                    \item private void startSignUp()
                    	Method which signs up new users to the application 
                    \item public void onComplete()
                    	Method used to check if the sign up process was sucessfull or not.
                    \item protected void onActivityResult()
                    	Method is called when the google place picker returns after the user has selected the restaurant of his choice, we use this to get the name and address of the place.
                    
                    	
                \end{itemize}
    \end{itemize}


\end{homeworkProblem}

\end{document}
